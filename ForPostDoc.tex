% !TeX program = lualatex

\documentclass[10pt,letterpaper,english]{moderncv} 
\moderncvstyle{banking} 

\definecolor{color0}{rgb}{0,0,0}% black
\definecolor{color1}{RGB}{224, 1, 34}
\definecolor{color2}{RGB}{0, 0, 0}

\usepackage[utf8]{inputenc}
\usepackage[top=.7in,left=.9in,bottom=.7in,right=.9in]{geometry}
\usepackage{babel}
\usepackage{tagging}
\usepackage{xparse}
\usepackage{tabularx} % exapandable columns
\usepackage{array} % Bold a column


% Tags:
%\usetag{details}
%\usetag{refs}
\usetag{physics}
\usetag{cern}


\firstname{Henry F.}
\familyname{Schreiner}
%\title{CV}
%\address{261 rue du Puits Mathieu}{Thoiry 01710 France}
\address{Department of Physics\enskip\textbullet\enskip University of Cincinnati}{PO Box 210011\enskip\textbullet\enskip Cincinnati, OH 45221-0011}

\phone{512-609-0314}
\email{hschrein@cern.ch}
\homepage{iscinumpy.gitlab.io}
%\homepage{homepages.uc.edu/\textasciitilde schreihf/uchenry}
\social[linkedin]{henryiii}
\social[github]{henryiii}
%\social[twitter]{henryschreiner3}

\NewDocumentCommand \mysection {m}{%
    \vspace{-1ex}
    \section{#1}
}

\begin{document}

\let\oldnullthing\null
\renewcommand{\null}{}
\makecvtitle
\renewcommand{\null}{\oldnullthing}
\vspace*{-10mm}

\mysection{Education}

\begin{tabularx}{\textwidth}{p{1.1in}X>{\bfseries}r}
	8/2010--5/2016 & Ph.D. in Experimental High Energy Physics &  University of Texas at Austin \\
	& \multicolumn{2}{l}{Dissertation: \textit{Methods and Simulations of Muon Tomography and Reconstruction}} \\[1ex]
%	& \multicolumn{2}{l}{Dissertation: \textit{From Images to Quantitative Understanding of Muon Tomography}} \\[1ex]
	1/2006--12/2009 & B.S. in Theoretical Physics, \textit{Summa Cum Laude} & Angelo State University \\
	& \textit{Distinguished Student for the College of Sciences} & \\
\end{tabularx}

\mysection{Career History}

\begin{tabularx}{\textwidth}{p{1.1in}X>{\bfseries}r}
	5/2016-- & Postdoctoral Fellow &  University of Cincinnati \\
	9/2018 & Consultant & University of Texas at Austin \\
	8/2018--12/2018 & Adjunct Instructor &  University of Cincinnati \\
	1/2011--5/2016 & Research Assistant & University of Texas at Austin \\
	6/2011--12/2015 & Head Teaching Assistant & University of Texas at Austin \\
	9/2010--5/2011 & Teaching Assistant & University of Texas at Austin \\
	1/2010--7/2010 & Special Projects Assistant & Angelo State University \\
	6/2008--9/2008 & Research Experience for Undergraduates & Northwestern University \\
\end{tabularx}

\subsection{Other research projects}

\begin{tabularx}{\textwidth}{p{1.1in}X>{\bfseries}r}
	2009 & Wavelet methods and their use in imaging and compression &  Angelo State University \\
	2007--2009 & Mapped the Dagger Mountain Anticline in Big Bend, TX & Angelo State University \\
	2007 & Characterization of near earth objects via orbital perturbations & Angelo State University \\
\end{tabularx}



\mysection{Programming Skills and Experience}
\begin{tabularx}{\textwidth}{>{\bfseries}p{1.1in}X}
Languages & Python \bullet\ C++ \bullet\ C \bullet\ CUDA \bullet\ Cython \bullet\ Matlab \bullet\ Lua \bullet\ JavaScript \\
Toolsets & ROOT \bullet\ GooFit \bullet\ Hydra \bullet\ Geant4 \bullet\ Boost \bullet\ HDF5 \bullet\ OpenMP \bullet\ MPI \bullet\ PyBind11 \bullet\ PyTorch \bullet\ TensorFlow\\
Toolchains & Git \bullet\ CMake \bullet\ Gitbook \bullet\ Sphinx \bullet\ Doxygen \bullet\ Hugo \bullet\ Nikola\\
Other & 2D and 3D technical artist, work featured by Casio \\
\end{tabularx}


\mysection{Open Source Contributions}

\subsection{Owner or maintainer}

\begin{tabularx}{\textwidth}{>{\bfseries}p{1.1in}Xr}
    GooFit & A C++ GPU and OpenMP library for fitting data, with Python bindings. & Maintainer \\
	Plumbum & Python shell pipelining, CLI, terminal color, and SSH tools &  Release manager \\
	CLI11 & A powerful command line parser for C++11 & Primary author \\
	SciKit-HEP & Author of DecayLanguage & Maintainer \\
\end{tabularx}

\subsection{Large contributions}
\begin{tabularx}{\textwidth}{>{\bfseries}p{1.1in}X}
	Minuit2        & Built standalone and several patches, added Windows support \\
	CMake          &  OpenMP support for macOS, Several CUDA inprovements  \\
	PyBind11       &  Several new features and better CUDA support      \\
	BeautifulHugo  &   Lots of new website features    \\
	HighFive       &    New concise syntax, bugfixes, std::array support  \\
	Nlohmann::JSON & Wrote version 3.3, adding GCC 4.8 support (previously considered impossible) \\
	Intel Parallel STL & Wrote CMake build system and added HomeBrew formula \\
\end{tabularx}

\subsection{LHCb contributions}
\textbf{%
	Gaudi \bullet\ 
	LHCb-Docker \bullet\ 
	StarterKit \bullet\ 
	LHCb Masterclass \bullet\ 
	DevelopKit
}

\subsection{Smaller contributions and bugfixes}
\textbf{%
	Uproot  \bullet\ 
	Uproot-methods \bullet\ 
	Awkward Array \bullet\ 
	ROOT \bullet\ 
	IPython \bullet\ 
	Jupyter \bullet\ 
	SciKit-build \bullet\ 
	Fast histogram \bullet\  
	VexCL \bullet\ 
	Probfit  \bullet\ 
	NumPythia \bullet\ 
	Eigen  \bullet\ 
    ROOTPy \bullet\ 
    Numba \bullet\  
    Google Test  \bullet\ 
    LMod \bullet\
    PipEnv \bullet\ 
    FmtLib
}


\mysection{Selected Presentations}

\begin{tabularx}{\textwidth}{p{.35in}X>{\bfseries}r}
	
	2018 & New approaches to PV reconstruction & LHCb Computing Workshop \\ % Chia, Italy
	2018 & iMinuit: Interactive Python wrapper around MINUIT2 & ROOT User's Workshop \\  % 
	2018 & GooFit 2.2 & CHEP \\ % Sofia, Bulgaria
	2018 & Invited lecture: Bindings in Python & PyHEP  \\
    2018 & Advancements in GooFit & DIANA/HEP \\
	2017 & CLI11 & DIANA/HEP \\
	2017 & GooFit 2.0 & ACAT \\
	2017 & GPUs in LHCb & DPF \\ % Chicago, IL
	2017 & GooFit: A case study & PEARC \\
	
	
	2015 & Measurement Over Large Solid Angle of Low Energy Cosmic Ray Muon Flux &  AGU Fall meeting \\
	 & (poster) \emph{Outstanding Student Paper Award} & \\
	 2013 & Lunch Seminar: Computing Environments & UT Austin \\
	 2009 & Edge Effects in the Use of Wavelets for Partial Image Reconstruction & Pi Mu Epsilon Mathfest \\
	 & \emph{Outstanding Student Exposition and Research in Applied Mathematics} & \\
	 
	 2009 & Invited lecture for Alpha Chi induction ceremony & Alpha Chi  \\
	 
	 2009 & Wavelet Matrix Completion Methods and Their Effects In Image Compression &  \\
	  & \emph{Amir Moez Award for Best Student Presentation in Mathematics} & Texas Academy of Sciences \\
	  & \emph{Abstract award} & \\
	  & \multicolumn{2}{l}{The Controversial Origin of the Dagger Mountain Anticline, Big Bend National Park, Texas}  \\
	  & \emph{Honorable mention} & \\
	  
	  %2009 & Three Mathematics ``slow pitch'' seminars over wavelets and cryptology & ASU \\
	  
	  2009 & Folding on the Flanks of the Southernmost Laramide Uplift, Big Bend Region, Texas & GSA \\[1em]
	  
	  2008 & Dagger Mountain, Big Bend National Park, West Texas Does Not Overlie a Laccolith & GSA \\[1em]
	 
	 2008 & Electronic Structure Calculations of NiTi Based Ternary Systems  & APS--Texas Section \\
	 & \emph{Student Presentation Award} & \\
	 
	 2007 & Characterization of near earth objects via orbital perturbations: A numerical study & APS \\
	 %http://meetings.aps.org/link/BAPS.2007.TSS07.SPS2.1
	 
\end{tabularx}


%{2009}{Mathematics Problem of the Month}{Angelo State University}{San Angelo, TX}{Winner}
%{2008}{Mathematical Contest in Modeling}{COMAP}{}{Honorable Mention}{Sudoku Generator Model}
%{2008}{REU}{Northwestern University}{Evanston, IL}{Certificate of Achievement}

%{2015, 2013}{UT Physics Open House}{University of Texas at Austin}{Austin, TX}{High Energy Physics Poster, 2015 Group Leader}
%{2009}{Student Research Showcase}{Angelo State University}{San Angelo, Texas}{4 posters over 3 different fields}
%{2008}{Quadrennial Congress}{Sigma Pi Sigma}{Fermilab, IL}{Poster}


\mysection{Selected Publications}

\begin{tabularx}{\textwidth}{p{.8in}X>{\bfseries}r}

2017 & SI2: SSE: Extending the Physics Reach of LHCb in Run 3 Using Machine Learning in the Real-Time Data Ingestion and Reduction System  \textit{(accepted)} & \href{https://nsf.gov/awardsearch/showAward?AWD_ID=1739772}{NSF \#1739772} \\ % 
2017 & GooFit 2.0 & ACAT 2017 \\ % http://inspirehep.net/record/1632171
2016 & Development of Muon Imaging Technology
for Archaeological and Geophysical Applications & FastTIMES \\
2013--2015 & PHY102M lab manual rewrite & University of Texas \\
2013 & Classified project report & University of Texas\\
2010 & Coauthor for Modern Physics lab manual rewrite & Angelo State University\\
\end{tabularx}
\href{http://inspirehep.net/search?ln=en&p=find+jy+2017+or+jy+2018+and+author+schreiner+and+author+sokoloff&of=hb&action_search=Search&sf=earliestdate&so=d}
{And 81 papers as part of the LHCb collaboration}

\mysection{Other Writings}

\begin{tabularx}{\textwidth}{>{\bfseries}p{1.45in}X}
DevelopKit & Developing software in LHCb’s Run 3 \\
Modern CMake & Using CMake for your own project \\
CLI11 Tutorial & Command line parsing made beautiful \\
GitBook Plugin: Term & A powerful terminal formatting plugin and example (used in other GitBooks) \\
UC ROOT Tutorial & Basic ROOT for HEP \\
GooFit 2torial & Using GooFit, writing GooFit \\
PHY102m videos & Video series for the 102m labs at UT \\
\end{tabularx}

\mysection{Selected Websites}

\begin{tabularx}{\textwidth}{>{\bfseries}p{1.45in}Xr}
ISciNumPy & My blog over programming & \url{iscinumpy.gitlab.io} \\
UCHenry & A site for UC and LHCb specific activities& \url{homepages.uc.edu/~schreihf/uchenry}  \\
GooFit & Introductory page to GooFit & \url{goofit.github.io} \\
MayaMuon & Public facing site for the MayaMuon project & \url{www.hep.utexas.edu/mayamuon} \\
UT PHY102M & Webpage for the intro Physics labs & \url{web2.ph.utexas.edu/~phy102m}  \\
\end{tabularx}

\end{document}
