% !TeX program = lualatex

\documentclass[10pt,letterpaper]{moderncv}
\moderncvstyle{banking}

\definecolor{color0}{rgb}{0,0,0}% black
\definecolor{color1}{RGB}{0, 0, 0} % was { 224, 1, 34}
\definecolor{color2}{RGB}{0, 0, 0}

\usepackage[T1]{fontenc}
\usepackage[utf8]{inputenc}
\usepackage[top=.7in,left=.85in,bottom=.7in,right=.85in]{geometry}
\usepackage{babel}
\usepackage{tagging}
\usepackage{xparse}
\usepackage{tabularx} % Expandable columns
\usepackage{array} % Bold a column



% Tags:
%\usetag{details}
%\usetag{refs}
\usetag{physics}
\usetag{cern}


\firstname{Henry F.}
\familyname{Schreiner}
%\title{CV}
%\address{261 rue du Puits Mathieu}{Thoiry 01710 France}
\address{Princeton Research Computing\enskip\textbullet\enskip Princeton University}{347 Lewis Science Library\enskip\textbullet\enskip Princeton, New Jersey 08544}
\phone{609-258-8141}
\email{hschrein@cern.ch}
\homepage{iscinumpy.gitlab.io}
\social[linkedin]{henryiii}
\social[github]{henryiii}
\social[twitter]{henryschreiner3}
\social[orcid]{0000-0002-7833-783X}

\NewDocumentCommand \mysection {m}{%
    \vspace{-1ex}
    \section{#1}
}

\begin{document}

\let\oldnullthing\null
\renewcommand{\null}{}
\makecvtitle
\renewcommand{\null}{\oldnullthing}
\vspace*{-10mm}

\mysection{Education}

\begin{tabularx}{\textwidth}{p{1.1in}X>{\bfseries}r}
	8/2010--5/2016 & Ph.D. in Experimental High Energy Physics &  University of Texas at Austin \\
	& \multicolumn{2}{l}{Dissertation: \textit{Methods and Simulations of Muon Tomography and Reconstruction}} \\[1ex]
	1/2006--12/2009 & B.S. in Theoretical Physics, \textit{Summa Cum Laude} & Angelo State University \\
	& \textit{Distinguished Student for the College of Sciences} & \\
\end{tabularx}

\mysection{Career history}

\begin{tabularx}{\textwidth}{p{1.1in}X>{\bfseries}r}
    3/2019-- & Computational Physicist/Research Software Engineer & Princeton University \\
	5/2016--2/2019 & Postdoctoral Fellow &  University of Cincinnati \\
	9/2018 & Consultant & University of Texas at Austin \\
	8/2018--12/2018 & Adjunct Instructor &  University of Cincinnati \\
	1/2011--5/2016 & Research Assistant & University of Texas at Austin \\
	6/2011--12/2015 & Head Teaching Assistant & University of Texas at Austin \\
	9/2010--5/2011 & Teaching Assistant & University of Texas at Austin \\
	1/2010--7/2010 & Special Projects Assistant & Angelo State University \\
	6/2008--9/2008 & Research Experience for Undergraduates & Northwestern University \\
\end{tabularx}

\begin{comment}
\subsection{Other research projects}

\begin{tabularx}{\textwidth}{p{1.1in}X>{\bfseries}r}
	2009 & Wavelet methods and their use in imaging and compression &  Angelo State University \\
	2007--2009 & Mapped the Dagger Mountain Anticline in Big Bend, TX & Angelo State University \\
	2007 & Characterization of near earth objects via orbital perturbations & Angelo State University \\
\end{tabularx}
\end{comment}

\mysection{Classes taught}
\begin{tabularx}{\textwidth}{p{.8in}X>{\bfseries}r}
    Fall 2024 & APC 524/MAE 506/AST 506: \textit{Software Engineering for Scientific Computing} & Princeton University \\
    Spring 2024 & Se4Sci: \textit{Software Engineering for Scientific Computing}---Virtual & IRIS-HEP \\
    Fall 2023 & APC 524/MAE 506/AST 506: \textit{Software Engineering for Scientific Computing} & Princeton University \\
    Fall 2022 & APC 524/MAE 506/AST 506: \textit{Software Engineering for Scientific Computing} & Princeton University \\
    Fall 2018 & PHYS 5041/6041: \textit{Computational Physics} & University of Cincinnati \\
\end{tabularx}

\mysection{Grants}
\begin{tabularx}{\textwidth}{p{.5in}X>{\bfseries}r}
    2022 & Elements: Simplifying Compiled Python Packaging in the Sciences (PI) & \href{https://nsf.gov/awardsearch/showAward?AWD_ID=NSF 2209877}{NSF \#2209877} \\
    2017 & SI2\@: SSE\@: Extending the Physics Reach of LHCb in Run 3 Using Machine Learning in the Real-Time Data Ingestion and Reduction System & \href{https://nsf.gov/awardsearch/showAward?AWD_ID=1739772}{NSF \#1739772} \\ %
\end{tabularx}

\mysection{Selected publications---peer reviewed}

\begin{minipage}[t]{.06\textwidth}
\textbf{2024:}
\end{minipage}%
\begin{minipage}[t]{.94\textwidth}
\textbf{SciPy 2024} \emph{Scikit-build-core} Proceedings of the 23rd Python in Science Conference, 225--235
\end{minipage}

\begin{minipage}[t]{.06\textwidth}
\textbf{2023:}
\end{minipage}%
\begin{minipage}[t]{.94\textwidth}
\textbf{CHEP 2023} \emph{Awkward Just-In-Time (JIT) Compilation: A Developer’s Experience} EPJ Web of Conf., 295 (2024) \\
\textbf{CHEP 2023} \emph{Advances in developing deep neural networks for finding primary vertices in proton-proton collisions at the LHC} EPJ Web of Conf., 295 (2024)
\end{minipage}


\begin{minipage}[t]{.06\textwidth}
\textbf{2022:}
\end{minipage}%
\begin{minipage}[t]{.94\textwidth}
\textbf{ACAT 2022} \emph{Comparing and improving hybrid deep learning algorithms for identifying and locating primary vertices} \\
\textbf{SciPy 2022} \emph{Awkward Packaging: building Scikit-HEP} Proceedings of the 21th Python in Science Conference, 115--120
\end{minipage}

\begin{minipage}[t]{.06\textwidth}
\textbf{2021:}
\end{minipage}%
\begin{minipage}[t]{.94\textwidth}
\textbf{Article} \emph{Software Training in HEP} Comput Softw Big Sci \textbf{5}, Article number: 22 (2021) \\
\textbf{vCHEP 2021} \emph{Progress in developing a hybrid deep learning algorithm for identifying and locating primary vertices} EPJ Web Conf. 251 (2021) 04012
\end{minipage}

\begin{minipage}[t]{.06\textwidth}
\textbf{2020:}
\end{minipage}%
\begin{minipage}[t]{.94\textwidth}
\textbf{SciPy 2020} \emph{Boost-histogram: High-Performance Histograms as Objects} Proceedings of the 19th Python in Science Conference, 63--69 \\
\textbf{CtD 2020}   \emph{An updated hybrid DL algorithm for identifying and locating primary vertices} PROC-CTD2020-52 \\
\textbf{CHEP 2019}  \emph{Recent developments in histogram libraries} EPJ Web Conf. 245 (2020) 05014 \\
\textbf{CHEP 2019}  \emph{The Scikit HEP Project---overview and prospects} EPJ Web Conf. 245 (2020) 06028
\end{minipage}

\begin{minipage}[t]{.06\textwidth}
\textbf{2019:}
\end{minipage}%
\begin{minipage}[t]{.94\textwidth}
\textbf{ACAT 2019} \emph{A hybrid deep learning approach to vertexing} J. Phys.: Conf. Ser. 1525 012079 \\
\textbf{Article}   \emph{A comprehensive real-time analysis model at the LHCb experiment} JINST 14, no.04, P04006 \\
\textbf{CHEP 2018} \emph{A Python upgrade to the GooFit package for parallel fitting} EPJ Web Conf., 214 (2019) 05006
\end{minipage}

\begin{minipage}[t]{.06\textwidth}
\textbf{2018:}
\end{minipage}%
\begin{minipage}[t]{.94\textwidth}
\textbf{ACAT 2018} \emph{GooFit 2.0} ACAT 2017 J. Phys.: Conf. Ser. 1085 042014
\end{minipage}

\begin{minipage}[t]{.06\textwidth}
\textbf{2017:}
\end{minipage}%
\begin{minipage}[t]{.94\textwidth}
\textbf{PEARC17} \emph{Modernizing GooFit: A Case Study} In Proceedings of the Practice and Experience in Advanced Research Computing 2017 on Sustainability, Success and Impact
\end{minipage}

\mysection{Selected publications - other}

\begin{tabularx}{\textwidth}{p{.5in}X>{\bfseries}r}
2022 & Second Analysis Ecosystem Workshop Report & CERN \\
2021 & Software Training in HEP & vCHEP 2021 \\
2020 & Inexpensive multi-patient respiratory monitoring system for helmet ventilation during COVID-19 pandemic & \href{https://doi.org/10.1101/2020.06.29.20141283}{medRxiv} \\
2016 & Development of Muon Imaging Technology
for Archaeological and Geophysical Applications & FastTIMES \\
%
\end{tabularx}

\begin{taggedblock}{details}
\begin{tabularx}{\textwidth}{p{.8in}X>{\bfseries}r}
2013--2015 & PHY102M lab manual rewrite & University of Texas \\
2013 & Classified project report & University of Texas\\
2010 & Coauthor for Modern Physics lab manual rewrite & Angelo State University\\
\end{tabularx}
\end{taggedblock}


\mysection{Programming skills and experience}
\begin{tabularx}{\textwidth}{>{\bfseries}p{1.1in}X}
Languages & Python \textbullet\ C++ \textbullet\ Rust \textbullet\ Ruby \textbullet\ C \textbullet\ CUDA \textbullet\ Matlab \textbullet\ Lua \textbullet\ JavaScript
\end{tabularx}


\mysection{Open Source contributions}


\subsection{Selected projects, owner or maintainer}

\begin{tabularx}{\textwidth}{>{\bfseries}p{1.2in}Xr}
    pypa/build & The build tool for Python packages (50M dpm) & Maintainer \\
    pipx & Python application installer (4.4M dpm) & Co-maintainer \\
    cibuildwheel & Builds binary wheels on CI systems & Co-maintainer \\
    pyproject-metadata & Tool for PEP 621 backends (2.3M dpm) & Co-maintainer \\
    validate-pyproject & Validates Python configuration files & Co-maintainer \\
    dependency-groups & Tool for PEP 735 dependency groups & Co-maintainer \\
    pybind11 & The popular C++11 Python bindings (8.8M dpm) & Maintainer \\
    Nox & Task automation tool (2.2M dpm) & Co-maintainer \\
    Scikit-build & KitWare's CMake adaptor for Python & Admin \\
    Scikit-build-core & Modern CMake build system for Python (1.5M dpm) & Author \\
    meson-python & The meson adaptor for Python & Co-maintainer \\
    CMake & KitWare's Python packaging for CMake (7M dpm) & Maintainer \\
    Ninja & KitWare's Python packaging for Ninja (9M dpm) & Maintainer \\
    Scikit-HEP & A collection of packages for High Energy Physics & Admin \\
    Cookie & A template for new scientific-python projects & Primary author \\
    repo-review & WebAssembly \& CLI framework for checking repo health & Primary author \\
    sp-repo-review & Checks for repo-review for scientific-python & Primary author \\
    boost-histogram & Fast and comprehensive histograms & Primary author \\
    hist, uhi & Scikit-HEP packages for advanced histograms & Primary author \\
    uproot-browser & ROOT file graphical browser in the terminal & Primary author \\
	Particle & Scikit-HEP package presenting a particle database & An author \\
	DecayLanguage & Scikit-HEP package describing decay chains & An author \\
    GooFit & A C++ GPU \& OpenMP fitting library with Python bindings & Co-maintainer \\
    Plumbum & Python shell pipelining, CLI, color, and SSH tools (4.3M dpm) & Maintainer \\
	CLI11 & A command line parser for C++11 used by the Microsoft Terminal & Primary author \\
	ROOT & High energy physics C++ toolkit & conda-forge and homebrew \\
	Numba & Provides Numba on Conda & conda-forge maintainer \\
%	trampolim & Modern pure Python package system & Co-maintainer \\
    jekyll-indico & Ruby plugin for Jekyll to collect Indico talks & Primary author \\
    hypernewsviewer & Replacement for CMS HyperNews & Primary author \\
    check-sdist & Tool for checking Python packaging & Primary author \\
\end{tabularx}

\subsection{Other contributions}
\begin{tabularx}{\textwidth}{>{\bfseries}p{1.4in}X}
	CMake          & OpenMP support for macOS, an occasional CUDA maintainer  \\
	HighFive       & New concise syntax, bugfixes, std::array support  \\
	Intel Parallel STL & Wrote CMake build system and added HomeBrew formula \\
	Minuit2        & Built standalone and several patches, added Windows support \\
	Nlohmann::JSON & Wrote version 3.3, adding GCC 4.8 support (previously considered impossible) \\
    Manylinux      & Assisted with CI automation and user experience improvements \\
    Pyodide        & Multiple checks and fixes, updated most packages, added boost-histogram \\
    Wheel          & Tags tool \& new backend without circular dependencies \\
\end{tabularx}

\subsection{LHCb contributions}
\textbf{%
	Gaudi \textbullet\
	LHCb-Docker \textbullet\
	StarterKit \textbullet\
	LHCb Masterclass \textbullet\
	DevelopKit
}


\begin{taggedblock}{details}
\subsection{Smaller contributions and bugfixes}
\textbf{%
	Awkward Array \textbullet\
	Eigen \textbullet\
	Fast histogram \textbullet\
	IPython \textbullet\
	Jupyter \textbullet\
	NumPythia \textbullet\
	Probfit \textbullet\
	ROOT \textbullet\
	Scikit-build \textbullet\
	Uproot \textbullet\
	Uproot-methods \textbullet\
	VexCL \textbullet\
    FmtLib \textbullet\
    Google Test \textbullet\
    LMod \textbullet\
    MPLHEP \textbullet\
    ManyLinux \textbullet\
    Numba \textbullet\
    Packaging \textbullet\
    Pip \textbullet\
    PipEnv \textbullet\
    Pixi \textbullet\
    Poetry \textbullet\
    Probfit \textbullet\
    PyJet \textbullet\
    Python \textbullet\
    ROOTPy \textbullet\
    Rich \textbullet\
    Ruff \textbullet\
    Ruff \textbullet\
    Setuptools-scm \textbullet\
    Typeshed
}
\end{taggedblock}


\mysection{Selected accomplishments}

\textbf{2025:}
Developing pybind11 3.0 and cibuildwheel 3.0 packages \textbullet\
Developed a new nox version with many long-requested features \textbullet\


\textbf{2024:}
Added support for Python 3.13 and free-threading dozens of packages \textbullet\
Implemented NumPy 2 support in multiple packages \textbullet\
Implemented PEP 639 in scikit-build-core, pyproject-metadata, and validate-pyproject \textbullet\
Implemented PEP 735 in nox, validate-pyproject, and cibuildwheel \textbullet\
Scikit-build-core reached 2.5 million downloads per month \textbullet\
Solved a multi-year fight between multipart and python-multipart \textbullet\
check-sdist 1.0 released \textbullet\
Worked on new versions of many major packages, including nox, pybind11,
cibuildwheel, build, validate-pyproject, CLI11, scikit-build-core, boost-histogram/hist,
plumbum, and more \textbullet\
Started a vote that got dependency-groups accepted into the PyPA \textbullet\
Joined the nox project.

\textbf{2023:}
wheel 0.40 released with ``tags'' interface (after 17 months!) \textbullet\
Setup scientific-python development guide and new package template \textbullet\
Released repo-review as part of scientific-python \textbullet\
Continued success with scikit-build-core, including being the recommendation for nanobind \textbullet\
Moved dozens of packages to Ruff \textbullet\
Completed the 2023 Advent of Code in Rust \textbullet\
Implemented support for PEP 723 in pipx \textbullet\
Developed the INTERSECT packaging curriculum.

\textbf{2022:}
Multiple talks at PyCon US 2022 \textbullet\
Two talks and two panels at SciPy 22 \textbullet\
Released uproot-browser \textbullet\
Featured as Python Dev of the Week on MouseVsPython \textbullet\
Compiled boost-histogram in WebAssembly, in the next pyodide release \textbullet\
Scikit-build-core introduced and used for Awkward 2.0 \textbullet\
Taught \emph{Level Up Your Python} during Wintersession.

\textbf{2021:}
Featured on the TalkPython Podcast \#339 \textbullet\
Awarded Google Open Source Peer Bonus Award \textbullet\
Joined the PyPA \textbullet\ Rewrote portions of packaging.python.org.\ \textbullet\
Released boost-histogram 1.x series, UHI, and hist 2.2--2.5\ \textbullet\
Joined the ROOT project.\ \textbullet\
Joined the Scikit-build project as an admin \textbullet\
Built new config mode for cibuildwheel 2.0 \textbullet\
Mentored three fellowship students.

\begin{taggedblock}{details}
\textbf{2020:}
Developed software for COVID-19 flow monitor \textbullet\
Joined and released pybind11 2.6.x series \textbullet\
Joined cibuildwheel \textbullet\
Released cookie and major updates to Developer docs \textbullet\
Proposed PlottableHistogram Protocol and started UHI \textbullet\
Released jekyll-indico for Ruby \textbullet\
CMake tutorial accepted into HSF \textbullet\
GitHub Actions conversions and Scikit-HEP developer docs \textbullet\
Successful Google Summer of Code mentor.

\textbf{2019:}
Co-created Conda-forge: ROOT \textbullet\
Joined Princeton \textbullet\
Proposed and started boost-histogram \textbullet\
Redesigned Scikit-hep.org, started the developer pages \textbullet\
Took over IRIS-HEP.org development \textbullet\
Fixed multiple Scikit-HEP legacy packages \textbullet\ Created CMake workshop.

\textbf{2018:} New standalone Minuit2 for ROOT \textbullet\
New GooFit releases \textbullet\
Joined Scikit-HEP \textbullet\
Taught graduate course on computational physics \textbullet\
Co-developed pv-finder algorithm.
\end{taggedblock}

\mysection{Other writings}

\begin{tabularx}{\textwidth}{>{\bfseries}p{1.6in}X}
Level Up Your Python & Designed for PRC tutorial \\
CMake workshop & Designed for the USATLAS/FIRST-HEP computing bootcamp \\
CompClass & A full semester over Computational Science in Python \\
DevelopKit & Developing software in LHCb's Run 3 \\
Modern CMake & Using CMake for your own project \\
CLI11 Tutorial & Command line parsing made beautiful \\
GitBook Plugin: Term & A powerful terminal formatting plugin and example (used in other GitBooks) \\
UC ROOT Tutorial & Basic ROOT for HEP \\
GooFit 2torial & Using GooFit, writing GooFit \\
PHY102m videos & Video series for the 102m labs at UT \\
\end{tabularx}

\mysection{Selected websites}

\begin{tabularx}{\textwidth}{>{\bfseries}p{1.45in}Xr}
clariphy & Community Laboratory for AI Research at the Intersection with Physics & \url{clariphy.org} \\
science-responds & The Larger Scientific Community Responds to the COVID-19 Pandemic & \url{science-responds.org} \\
ISciNumPy & My blog over programming & \url{iscinumpy.gitlab.io} \\
IRIS-HEP & Home of the IRIS-HEP project & \url{iris-hep.org} \\
Scikit-HEP & Home of the Scikit-HEP project & \url{scikit-hep.org} \\
Scientific-Python Development Guide & Guide on developing with Python & \url{learn.scientific-python.org} \\
UCHenry & A old site for UC and LHCb specific activities & \url{cern.ch/hschrein}  \\
MayaMuon & Public facing site for the MayaMuon project & \url{www.hep.utexas.edu/mayamuon} \\
UT PHY102M & Webpage for the intro Physics labs & \url{web2.ph.utexas.edu/~phy102m}  \\
\end{tabularx}


\mysection{Selected presentations}

\begin{tabularx}{\textwidth}{p{.33in}X>{\bfseries}r}
    2025 & Building Binary Packages & Princeton RSE group \\
    2025 & Tools to help you write better code & Princeton Wintersession \\
    2024 & The two flavors of Python 3.13 & PyHEP 2024 \\
    2024 & Modern binary build systems & PyCon 2024 \\
    2024 & Learning Rust with Advent of Code 2023 & Princeton RSE group \\
    2023 & Projects report & Princeston RSE Peer Network \\
    2023 & Software Quality Assurance Tooling & PrincetonPy \\
    2023 & Scientific-Python developer summit report & Princeton RSE group \\
    2023 & Awkward Just-In-Time (JIT) Compilation & CHEP 2023 \\
    2023 & Analysis of physics analysis & CHEP 2023 \\
    2023 & Dynamic Metadata proposal (packaging summit) & PyCon 23 \\
    2022 & Everything you didn't know you needed & CoDaS HEP 2022 \\
    2022 & Analysis Grand Challenge / HEP SP Ecosystem & DANCE/CoDaS 2022 \\
    2022 & Building Binary Extensions & SciPy '22 \\
    2022 & Awkward Packaging & SciPy '22 \\
    2022 & Digital RSE: Automatic Code Quality Checks & Princeton RSE group \\
    2023 & Building a Binary Extension & PyCon 22 \\
    2023 & Extensionlib proposal & PyCon 22 (packaging summit) \\
    2023 & Guidelines for modern packaging (lightning talk) & PyCon 22 \\
    2022 & boost-histogram/Hist & PyHEP 2022 Topical \\
    2022 & CMake best practices & HSF S\&C round table \\
	2021 & Powerful Python Packaging for Scientific Codes & PyHEP 2021 \\
	2021 & High-Performance Histogramming for HEP Analysis & PyHEP 2021 \\
	2021 & Level Up Your Python & PyHEP 2021 \\
	2021 & Pybind11 & SciPy '21 \\
    2021 & CMake: Best Practices & S\&C Round Table \\
    2021 & Level Up Your Python & PRC Bootcamp \\
    2020 & High-performance Python: CPUs & PRC Training \\
    2020 & Mixing Python and Compiled Code & PRC Training \\
    2020 & Building a Python library & Princeton RSE group \\
    2020 & High-performance Python: CPUs & PRC Training Series \\
    2020 & The boost-histogram package & PyHEP 2020 \\
    2020 & Boost-histogram: High-Performance Histograms as Objects & SciPy '20 \\
    2020 & Boost-Histogram for Analysis Systems (poster) & IRIS-HEP Poster Session \\
    2020 & International efforts on efficient computing & ECHEP \\
    2020 & Python, NumPy, and Pandas & Princeton RDM 2020 \\
    2019 & High performance Python: GPUs & PRC Minicourses \\
    2019 & High performance Python: CPUs & PRC Minicourses \\
    2019 & Recent developments in histogram libraries  & CHEP 2019 \\
    2019 & Python 3.8: What's new & PyHEP 2019 \\
    2019 & Python Histogramming packages & PyHEP 2019\\
    % 2019 & Modern CMake Workshop & USATLAS Bootcamp \\
    2019 & A hybrid deep learning approach to vertexing & 3rd IML Workshop \\
    2019 & A hybrid deep learning approach to vertexing & CtD/WIT 2019 \\
    2019 & Conda: a complete reproducible ROOT environment in under 5 minutes & HOW 2019 \\
    2019 & Machine Learning for the Primary Vertex reconstruction & HOW 2019 \\
    2019 & A hybrid deep learning approach to vertexing & ACAT 2019 \\ %  Saas Fee, Switzerland
	2018 & New approaches to PV reconstruction & LHCb Computing Workshop \\ % Chia, Italy
	2018 & iMinuit: Interactive Python wrapper around MINUIT2 & ROOT User's Workshop \\  %
	2018 & GooFit 2.2 & CHEP \\ % Sofia, Bulgaria
	2018 & Invited lecture: Bindings in Python & PyHEP  \\
    2018 & Advancements in GooFit & DIANA/HEP \\
	2017 & CLI11 & DIANA/HEP \\
	2017 & GooFit 2.0 & ACAT \\
	2017 & GPUs in LHCb & DPF \\ % Chicago, IL
	2017 & GooFit: A case study & PEARC \\
\end{tabularx}

\begin{taggedblock}{details}
\begin{tabularx}{\textwidth}{p{.33in}X>{\bfseries}r}

	2015 & Measurement Over Large Solid Angle of Low Energy Cosmic Ray Muon Flux &  AGU Fall meeting \\
	 & (poster) \emph{Outstanding Student Paper Award} & \\
	 2013 & Lunch Seminar: Computing Environments & UT Austin \\
	 2009 & Edge Effects in the Use of Wavelets for Partial Image Reconstruction & Pi Mu Epsilon Mathfest \\
	 & \emph{Outstanding Student Exposition and Research in Applied Mathematics} & \\

	 2009 & Invited lecture for Alpha Chi induction ceremony & Alpha Chi  \\

	 2009 & Wavelet Matrix Completion Methods and Their Effects In Image Compression &  \\
	  & \emph{Amir Moez Award for Best Student Presentation in Mathematics} & Texas AoS \\
	  & \emph{Abstract award} & \\

	  2009 & The Controversial Origin of the Dagger Mountain Anticline, Big Bend National Park, Texas &  Texas AoS \\
	  & \emph{Honorable mention} & \\

	  %2009 & Three Mathematics ``slow pitch'' seminars over wavelets and cryptology & ASU \\

	  2009 & Folding on the Flanks of the Southernmost Laramide Uplift, Big Bend Region, Texas & GSA \\

	 % 2008 & Dagger Mountain, Big Bend National Park, West Texas Does Not Overlie a Laccolith & GSA \\

	 2008 & Electronic Structure Calculations of NiTi Based Ternary Systems  & APS--Texas Section \\
	 & \emph{Student Presentation Award} & \\

	 2007 & Characterization of near earth objects via orbital perturbations: A numerical study & APS
	 %http://meetings.aps.org/link/BAPS.2007.TSS07.SPS2.1

\end{tabularx}
\end{taggedblock}



%{2009}{Mathematics Problem of the Month}{Angelo State University}{San Angelo, TX}{Winner}
%{2008}{Mathematical Contest in Modeling}{COMAP}{}{Honorable Mention}{Sudoku Generator Model}
%{2008}{REU}{Northwestern University}{Evanston, IL}{Certificate of Achievement}

%{2015, 2013}{UT Physics Open House}{University of Texas at Austin}{Austin, TX}{High Energy Physics Poster, 2015 Group Leader}
%{2009}{Student Research Showcase}{Angelo State University}{San Angelo, Texas}{4 posters over 3 different fields}
%{2008}{Quadrennial Congress}{Sigma Pi Sigma}{Fermilab, IL}{Poster}



\end{document}
